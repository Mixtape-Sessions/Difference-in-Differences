\documentclass{beamer}

\input{preamble.tex}
\usepackage{breqn} % Breaks lines

\usepackage{amsmath}
\usepackage{mathtools}

\usepackage{pdfpages} % \includepdf

\usepackage{listings} % R code
\usepackage{verbatim} % verbatim

% Video stuff
\usepackage{media9}

% packages for bibs and cites
\usepackage{natbib}
\usepackage{har2nat}
\newcommand{\possessivecite}[1]{\citeauthor{#1}'s \citeyearpar{#1}}
\usepackage{breakcites}
\usepackage{alltt}

% Setup math operators
\DeclareMathOperator{\E}{E} \DeclareMathOperator{\tr}{tr} \DeclareMathOperator{\se}{se} \DeclareMathOperator{\I}{I} \DeclareMathOperator{\sign}{sign} \DeclareMathOperator{\supp}{supp} \DeclareMathOperator{\plim}{plim}
\DeclareMathOperator*{\dlim}{\mathnormal{d}\mkern2mu-lim}
\newcommand\independent{\protect\mathpalette{\protect\independenT}{\perp}}
   \def\independenT#1#2{\mathrel{\rlap{$#1#2$}\mkern2mu{#1#2}}}
\newcommand*\colvec[1]{\begin{pmatrix}#1\end{pmatrix}}

\newcommand{\myurlshort}[2]{\href{#1}{\textcolor{gray}{\textsf{#2}}}}


\begin{document}

\imageframe{./lecture_includes/mixtape_did_cover.png}


% ---- Content ----

\section{Basics}

\subsection{Outline}

\begin{frame}{Workshop outline}

\begin{itemize}
\item Introduction to DiD basics 
	\begin{itemize}
	\item Potential outcomes review
	\item DiD formula
	\item Covariates
	\end{itemize}
\item Differential timing
	\begin{itemize}
	\item Heterogeneity
	\item TWFE bias in estimation of overall and dynamic ATT
	\end{itemize}
\end{itemize}

\end{frame}

\begin{frame}{Workshop outline}

\begin{itemize}
\item Three types of solutions
	\begin{itemize}
	\item Aggregated group-time ATT
	\item Stacked regression
	\item Explicit Imputation
	\end{itemize}
\item Continuous treatments
\item Fuzzy difference-in-differences
\end{itemize}

\end{frame}

\begin{frame}{What is difference-in-differences (DiD)}

\begin{itemize}
\item DiD is a very old, relatively straightforward, intuitive research design
\item A group of units are assigned some treatment and then compared to a group of units that weren't
\item Early usage in several 19th century health policy debates 
\item Brought into labor economics with Orley Ashenfelter (1978), LaLonde (1986), Card and Krueger (1994) 
\item Now the most widely used quasi-experimental method
\end{itemize}

\end{frame}


\imageframe{./lecture_includes/currie_2.jpg}

\begin{frame}{Why an entire workshop on DiD?}

\begin{itemize}
\item \textbf{Research advantages}: DiD is often one of the only ways to study large social policies (e.g., decriminalized prostitution [Cunningham and Shah 2018])
\item \textbf{Worrisome news}: Many new papers suggest canonical methods are biased, maybe severely biased
\item \textbf{Good news}: Many new solutions and widely available code in both R and Stata
\item \textbf{Econometrics}:  It's always fun to learn econometrics
\end{itemize}

\end{frame}

\begin{frame}{Pedagogy of the seminar}

\begin{itemize}
\item Emphasis on assumptions and authors
\item It can feel like drinking from a firehose to learn so many papers
\item I can't really advise you on how these are connected to one another, as that level of depth I'm still working on myself
\end{itemize}

\end{frame}


\subsection{Potential outcomes}

\begin{frame}{Potential outcomes review}

\begin{itemize}
\item DiD really can't be understood without committing to some common causality language
\item Standard language is the potential outcomes model, sometimes called the Rubin-Neyman model
\item Don't go over potential outcomes too fast or you'll miss all the fun
\item Potential outcomes are thought experiments about worlds that never existed, but which \emph{could have}
\end{itemize}

\end{frame}

\begin{frame}{Introduction to Counterfactuals and Causality}
	
	\begin{itemize}
	\item Aliens come and orbit earth, see sick people in hospitals and conclude ``these `hospitals' are hurting people'' 
	\item Motivated by anger and compassion, they kill the doctors to save the patients
	\item Sounds stupid, but earthlings do this too - all the time
	\item Let's look at the challenges of making causality synonymous with correlations
	\end{itemize}		
		
\end{frame}

\begin{frame}{\#1: Correlation and causality are very different concepts}

These are not the same thing:

		\begin{itemize}
	\item Causal question: ``If a doctor puts a person with Covid on a ventilator (D), will her health (Y) improve?''
	\item Correlation question:  $$\frac{Cov(D,Y)}{\sqrt{Var_D}\sqrt{{Var_Y}}}$$
		\end{itemize}

\end{frame}


\begin{frame}{\#2: Coming first may not mean causality!}

\begin{itemize}
\item Every morning the rooster crows and then the sun rises
\item If the feral cat had killed the rooster the sun would have still risen, so coming first must not be enough
\item \emph{Post hoc ergo propter hoc}: ``after this, therefore, because of this''
\end{itemize}

\end{frame}


\imageframe{./lecture_includes/scottboat.jpg}


\begin{frame}{\#3: No correlation does not mean no causality!}

\begin{itemize}
	\item A sailor sails her sailboat across a lake
	\item Wind blows, and she perfectly counters by turning the rudder
	\item The same aliens observe from space and say ``Look at the way she's moving that rudder back and forth but going in a straight line.  That rudder is broken.'' So they send her a new rudder
	\item They're wrong but why are they wrong? There is, after all, no correlation
	\item Question: What if she had been moving the rudder by flipping coins?
\end{itemize}

\end{frame}



\begin{frame}{Potential outcomes notation}
	
	\begin{itemize}
	\item Let the treatment be a binary variable: $$D_{i,t} =\begin{cases} 1 \text{ if hospitalized at time $t$} \\ 0 \text{ if not hospitalized at time $t$} \end{cases}$$where $i$ indexes an individual observation, such as a person

	\item Potential outcomes: $$Y_{i,t}^j =\begin{cases} 1 \text{ health if hospitalized at time $t$} \\ 0 \text{ health if not hospitalized at time $t$} \end{cases}$$where $j$ indexes a counterfactual state of the world

	\item I'll drop $t$ subscript, but note -- these are potential outcomes for the same person at the exact same moment in time
	\end{itemize}
\end{frame}

\begin{frame}{Moving between worlds}

\begin{itemize}
\item A potential outcome $Y^1$ and a historical outcome $Y$ are neither conceptually nor notationally the same thing
\item Potential outcomes are \emph{hypothetical} possibilities describing states of the world but historical outcomes actually occurred
\item We choose among potential outcomes by selecting the treatment
\end{itemize}
\end{frame}



\begin{frame}{Important definitions}
	\begin{block}{Definition 1: Individual treatment effect}
	    The individual treatment effect,  $\delta_i$, equals $Y_i^1-Y_i^0$
	\end{block}
\end{frame}

\begin{frame}{Important definitions}
	\begin{block}{Definition 2: Average treatment effect (ATE)}
        The average treatment effect is the population average of all $i$ individual treatment effects 
        \begin{eqnarray*}
        E[\delta_i]&=&E[Y_i^1-Y_i^0]\\
        &=&E[Y^1_i] - E[Y^0_i]
        \end{eqnarray*}
	\end{block}
\end{frame}

\begin{frame}{Important definitions}
	\begin{block}{Definition 3: Switching equation}
	    An individual's observed health outcomes, $Y$, is determined by treatment assignment, $D_i$, and corresponding potential outcomes:
		      \begin{eqnarray*}
	      Y_i& = D_iY^1_i+(1-D_i)Y^0_i& \\
	      Y_i& = \begin{cases}
	      		Y^1_i\text{ if }D_i=1 \\
			Y^0_i\text{ if }D_i=0
			\end{cases}
	    		\end{eqnarray*}
	\end{block}
\end{frame}

\begin{frame}{So what's the problem?}

	\begin{block}{Definition 4: Fundamental problem of causal inference}	
	If you need both potential outcomes to know causality with certainty, then since it is impossible to observe both $Y_i^1$ and $Y_i^0$ for the same individual, $\delta_i$, is \emph{unknowable}.
	\end{block}

\end{frame}


\begin{frame}{Conditional Average Treatment Effects}	
	\begin{block}{Definition 5: Average Treatment Effect on the Treated (ATT)}
	The average treatment effect on the treatment group is equal to the average treatment effect conditional on being a treatment group member:
		\begin{eqnarray*}
		E[\delta|D=1]&=&E[Y^1-Y^0|D=1] \nonumber \\
		&=&E[Y^1|D=1]-E[Y^0|D=1]
		\end{eqnarray*}
	\end{block}
\end{frame}

\begin{frame}{Conditional Average Treatment Effects}
	\begin{block}{Definition 6: Average Treatment Effect on the Untreated (ATU)}
	The average treatment effect on the untreated group is equal to the average treatment effect conditional on being untreated:
		\begin{eqnarray*}
		E[\delta|D=0]&=&E[Y^1-Y^0|D=0] \nonumber \\
		&=&E[Y^1|D=0]-E[Y^0|D=0]
		\end{eqnarray*}
	\end{block}

\end{frame}









\end{document}
