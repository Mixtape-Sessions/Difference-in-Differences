\documentclass{beamer}

\input{preamble.tex}
\usepackage{breqn} % Breaks lines

\usepackage{amsmath}
\usepackage{mathtools}

\usepackage{pdfpages} % \includepdf

\usepackage{listings} % R code
\usepackage{verbatim} % verbatim

% Video stuff
\usepackage{media9}

% packages for bibs and cites
\usepackage{natbib}
\usepackage{har2nat}
\newcommand{\possessivecite}[1]{\citeauthor{#1}'s \citeyearpar{#1}}
\usepackage{breakcites}
\usepackage{alltt}

% Setup math operators
\DeclareMathOperator{\E}{E} \DeclareMathOperator{\tr}{tr} \DeclareMathOperator{\se}{se} \DeclareMathOperator{\I}{I} \DeclareMathOperator{\sign}{sign} \DeclareMathOperator{\supp}{supp} \DeclareMathOperator{\plim}{plim}
\DeclareMathOperator*{\dlim}{\mathnormal{d}\mkern2mu-lim}
\newcommand\independent{\protect\mathpalette{\protect\independenT}{\perp}}
   \def\independenT#1#2{\mathrel{\rlap{$#1#2$}\mkern2mu{#1#2}}}
\newcommand*\colvec[1]{\begin{pmatrix}#1\end{pmatrix}}

\newcommand{\myurlshort}[2]{\href{#1}{\textcolor{gray}{\textsf{#2}}}}


\begin{document}

\imageframe{./lecture_includes/mixtape_did_cover.png}


% ---- Content ----


\section{Covariates}

\subsection{Simple case, no covariates}

\begin{frame}{John Snow and cholera}

\begin{itemize}
\item John Snow, epidemiologist in 19th century, usually credited with first use of DiD
\item Believed cholera was spread through the Thames water supply which contradicted dominant theory about ``dirty air'' transmission
\item Grand experiment: Lambeth moves its pipe between 1849 and 1854; Southwark and Vauxhall delay
\item How can he use this event to test his hypothesis? Three ways: simple comparisons, interrupted time series of the difference in differences (DiD)
\end{itemize}

\end{frame}





\begin{frame}{Simple cross-sectional design}

\begin{table}\centering
		\caption{Lambeth and Southwark and Vauxhall, 1854}
		\begin{center}
		\begin{tabular}{ll}
		\toprule
		\multicolumn{1}{l}{\textbf{Company}}&
		\multicolumn{1}{c}{\textbf{Cholera mortality}}\\
		\midrule
		Lambeth  & $Y=L + D$ \\
		\midrule
		Southwark and Vauxhall  & $Y=SV$ \\
		\bottomrule
		\end{tabular}
		\end{center}
	\end{table}

\bigskip

\begin{eqnarray*}
\widehat{\delta}_{cs} = D + (L-SV)
\end{eqnarray*}

\end{frame}

\begin{frame}{Interrupted time series design}

	\begin{table}\centering
		\caption{Lambeth, 1849 and 1854}
		\begin{center}
		\begin{tabular}{lll}
		\toprule
		\multicolumn{1}{l}{\textbf{Company}}&
		\multicolumn{1}{c}{\textbf{Time}}&
		\multicolumn{1}{c}{\textbf{Cholera mortality}}\\
		\midrule
		Lambeth & 1849 & $Y=L$ \\
		& 1854 & $Y=L + (T + D)$ \\
		\bottomrule
		\end{tabular}
		\end{center}
	\end{table}

\begin{eqnarray*}
\widehat{\delta}_{its} = D + T
\end{eqnarray*}


\end{frame}

\begin{frame}{Difference-in-differences}

\begin{table}\centering
		\caption{Lambeth and Southwark and Vauxhall, 1849 and 1854}
		\begin{center}
		\begin{tabular}{lll|lc}
		\toprule
		\multicolumn{1}{l}{\textbf{Companies}}&
		\multicolumn{1}{c}{\textbf{Time}}&
		\multicolumn{1}{c}{\textbf{Outcome}}&
		\multicolumn{1}{c}{$D_1$}&
		\multicolumn{1}{c}{$D_2$}\\
		\midrule
		Lambeth & Before & $Y=L$ \\
		& After & $Y=L + T_L + D$ & $T_L+D$\\
		\midrule
		& & & & $D$ \\
		\midrule
		Southwark and Vauxhall & Before & $Y=SV$ \\
		& After & $Y=SV + T_{SV}$ & $T_{SV}$\\
		\bottomrule
		\end{tabular}
		\end{center}
	\end{table}

\begin{eqnarray*}
\widehat{\delta}_{did} = D + (T_L - T_{SV})
\end{eqnarray*}

\end{frame}




\begin{frame}{Sample averages}

\begin{eqnarray*}
\widehat{\delta}^{2x2}_{kU} = \bigg ( \overline{y}_k^{post(k)} - \overline{y}_k^{pre(k)} \bigg ) - \bigg ( \overline{y}_U^{post(k)} - \overline{y}_U^{pre(k)} \bigg )
\end{eqnarray*}

\end{frame}

\begin{frame}{Population expectations}

\begin{eqnarray*}
\widehat{\delta}^{2x2}_{kU} = \bigg ( E[Y_k|Post] - E[Y_k|Pre] \bigg ) - \bigg ( E[Y_U | Post ] - E[ Y_U | Pre] \bigg) \\
\end{eqnarray*}

\end{frame}


\begin{frame}{Potential outcomes and the switching equation}

\begin{eqnarray*}
\widehat{\delta}^{2x2}_{kU} &=& \bigg ( \underbrace{E[Y^1_k|Post] - E[Y^0_k|Pre] \bigg ) - \bigg ( E[Y^0_U | Post ] - E[ Y^0_U | Pre]}_{\mathclap{\text{Switching equation}}} \bigg)  \\
&&+ \underbrace{\textcolor{red}{E[Y_k^0 |Post] - E[Y^0_k | Post]}}_{\mathclap{\text{Adding zero}}} 
\end{eqnarray*}

\end{frame}

\begin{frame}{Parallel trends bias}

\begin{eqnarray*}
\widehat{\delta}^{2x2}_{kU} &=& \underbrace{E[Y^1_k | Post] - \textcolor{red}{E[Y^0_k | Post]}}_{\mathclap{\text{ATT}}} \\
&& + \bigg [  \underbrace{\textcolor{red}{E[Y^0_k | Post]} - E[Y^0_k | Pre] \bigg ] - \bigg [ E[Y^0_U | Post] - E[Y_U^0 | Pre] }_{\mathclap{\text{Non-parallel trends bias in 2x2 case}}} \bigg ]
\end{eqnarray*}


\end{frame}



\begin{frame}{OLS Specification}
	
	\begin{itemize}
	\item Properly specified OLS model will also identify the ATT when there is only two groups and no covariates
	\item Often preferred because
		\begin{itemize}
		\item OLS estimates the ATT under parallel trends
		\item Easy to calculate the standard errors
		\item Easy to include multiple periods
		\end{itemize}
	\item But some issues emerge with differential timing, covariates and continuous treatments
	\end{itemize}
\end{frame}

\begin{frame}{Regression DiD - Card and Krueger}
	
	\begin{itemize}
	\item The equivalent regression includes time and group fixed effects:$$Y_{its} = \alpha + \gamma NJ_s + \lambda d_t + \delta (NJ \times d)_{st} + \varepsilon_{its}$$
		\begin{itemize}
		\item NJ is a dummy equal to 1 if the observation is from NJ
		\item d is a dummy equal to 1 if the observation is from November (the post period)
		\end{itemize}
	\item This equation takes the following values
		\begin{itemize}
		\item PA Pre: $\alpha$
		\item PA Post: $\alpha + \lambda$
		\item NJ Pre: $\alpha + \gamma$
		\item NJ Post: $\alpha + \gamma + \lambda + \delta$
		\end{itemize}
	\item DiD equation: (NJ Post - NJ Pre) - (PA Post - PA Pre) $= \delta$
	\end{itemize}
\end{frame}




\begin{frame}[plain]
	$$Y_{ist} = \alpha + \gamma NJ_s + \lambda d_t + \delta(NJ\times d)_{st} + \varepsilon_{ist}$$
	\begin{figure}
	\includegraphics[scale=0.90]{./lecture_includes/waldinger_dd_5.pdf}
	\end{figure}
\end{frame}

\begin{frame}{OLS with twoway fixed effects}


Under parallel trends, OLS estimates the ATT. Researchers often will use OLS with time-varying covariates, but this is not advised as it is only unbiased under more restrictive assumptions which we discuss next



\end{frame}


\subsection{IPW}


\begin{frame}


\bigskip
\begin{quote}
``A good way to do econometrics is to look for good natural experiments \emph{and} use statistical methods that can tidy up the confounding factors that nature has not controlled for us. -- Daniel McFadden
\end{quote}

\end{frame}


\begin{frame}{Inverse probability weighting DiD}

\begin{itemize}
\item Abadie (2005) proposed a DiD estimator that could incorporate covariates and get an unbiased estimate of the ATT
\item Researcher needs treatment and comparison group observed before and after treatment
\item If treatment group units are selected based on their (observed) covariates, then baseline covariates are also needed 
\item No randomization is needed;  just another version of parallel trends  called conditional parallel trends
\end{itemize}

\end{frame}


\begin{frame}{Time varying versus time invariant covariates}

\begin{itemize}
\item In a DiD, we may need to control for X because treatment is only conditional on X
\item But in TWFE, all time invariant covariates are absorbed by the unit fixed effects -- only time varying covariates will survive TWFE
\item But time varying covariates place restrictions, as we will see, on the DGP and run the threat of conditioning on outcomes if they were changed by the treatment
\item Abadie proposes using only the covariates at baseline to form weights in the simple DiD formula
\end{itemize}

\end{frame}

\begin{frame}{Three step method}

\begin{enumerate}
\item Compute each unit's ``after minus before'' which is the DD part
\item Then estimate a propensity score which you'll use to weight each unit
\item Finally, compare weighted changes in ``after minus before'' for treatment versus comparison groups
\end{enumerate}

\bigskip

You can have heterogeneous treatment effects, but not differential timing

\end{frame}

\begin{frame}{Terms}

\begin{itemize}
\item $t$ is year of treatment which doesn't vary across units (so no differential timing)
\item $Y^1$ and $Y^0$ are potential outcomes (counterfactual versus actual)
\item $D$ is 1 or 0 based on group and time
\item $b$ is the ``baseline'' which is similar to CS using $g$ as the one year pre-treatment
\item $X$ are ``baseline'' covariates \textbf{only} -- they do not vary over time, which means propensity scores are estimated off the $b$ period \textbf{only}
\end{itemize}

\end{frame}

\begin{frame}{Assumptions}

Kind of common for this propensity score literature to only have two assumptions.  But usually the first conditional independence.  Now it is parallel trends because this is DD

\begin{enumerate}
\item Conditional parallel trends $$E[Y^0_t - Y^0_b|D=1,X_b] - E[Y^0_t - Y^0_t | D=0, X_b]$$ (Notice the $b$ subscript.  What is that you think?)
\item Common support $$Pr(D=1)>0; Pr(D=1|X)<1$$ Let's see a picture of common support that I drew.  Apologies it's horrible
\end{enumerate}

\end{frame}

\begin{frame}{Trimming the propensity score to get common support}

	\begin{figure}
	\includegraphics[scale=0.05]{./lecture_includes/common_support_abadie.png}
	\end{figure}

\end{frame}

\begin{frame}{Definition and estimation}

Defining the ATT parameter of interest
\begin{equation}
ATT=E[Y^1_t - Y^0_t |D_t=1]
\end{equation}

\bigskip
Abadie's estimator
\begin{equation}
E\bigg [ \frac{Y_t - Y_b}{Pr(D_t=1)} \times \frac{D_t - Pr(D=1|X_b)}{1-Pr(D=1|X_b)} \bigg ]
\end{equation}

These are also using the ``Hajek'' (non-normalized) weights from the inverse probability weighting literature

\end{frame}


\begin{frame}{Propensity scores}

\begin{itemize}
\item Paper is titlted ``Semi-parametric DiD'' because Abadie imposes structure on the polynomials used to construct the propensity score
\item You can use OLS linear probability models or series logit estimation
\end{itemize}

\end{frame}

\begin{frame}{Estimating propensity scores}

It's common to hear people say that we don't know the propensity score; we can only estimate it. Same here -- we approximate it with regressions

\bigskip

\begin{equation}
\widehat{Pr}(X_b) = \widehat{\gamma_0} + \widehat{\gamma_1}X + \widehat{\gamma_2}X^2 + \dots \varepsilon
\end{equation}

\bigskip

\begin{equation}
\widehat{Pr}(X_b) =  F(\widehat{\gamma_0} + \widehat{\gamma_1}X + \widehat{\gamma_2}X^2 + \dots)
\end{equation}

\end{frame}


\begin{frame}{Stata}

Stata command is called absdid

\bigskip

You need treatment (varname), $X$ variables (can be a list), the order in which the variables occur (weird, but results change if the order changes), and the exact estimator (LPM or logit)

\bigskip

Why not try it yourselves using the LaLonde NSW job trainings program data?

\bigskip

\url{https://github.com/scunning1975/mixtape/raw/master/nsw_mixtape.dta}

\end{frame}

\subsection{DRDiD}

\begin{frame}{Doubly Robust Difference-in-differences}

\begin{itemize}
\item DR models control for covariates twice -- once using the propensity score, once using outcomes adjusted by regression -- and are unbiased so long as:
	\begin{itemize}
	\item The regression specification for the outcome is correctly specified
	\item The propensity score specification is correctly specified
	\end{itemize}
\item Sant'Anna and Zhao (2020) incorporated DR into DiD by combining inverse probability weighting and outcome regression into a single DiD model
\item It's in the engine of Callaway and Sant'Anna (2020) that we discuss later so it merits close study
\item One of my favorite lesser known of the new DiD papers
\end{itemize}

\end{frame}




\begin{frame}{Defining the target parameter -- the ATT}


\begin{eqnarray*}
\delta = E[Y^1_{it} - Y^0_{it} | D_i=1]
\end{eqnarray*}

\end{frame}

\begin{frame}{Basic assumptions of DiD}

Assumption 1: Assume panel data or repeated cross-sectional data

\bigskip

Handling repeated cross-sectional data is hairy, and so I've chosen to focus on the panel data for this talk, but results are similar for repeated cross sections

\end{frame}

\begin{frame}{Basic assumptions of DD}

Assumption 2: Conditional parallel trends

\bigskip

Counterfactual trends for the treatment group are the same as the control group for all values of $X$

\begin{eqnarray*}
E[Y_1^0 - Y_0^0 | X, D=1] = E[Y^0_1 - Y^0_0 | X, D=0]
\end{eqnarray*}

\end{frame}

\begin{frame}{Basic assumptions of DD}

Assumption 3: Common support or overlap

\bigskip

For some $e>0$, the probability of being in the treatment group is greater than $e$ and the probability of being in the treatment group conditional on $X$ is $\leq1-e$. 

\bigskip

Intuition of assumption 3: Called overlap or common support. Means there is at least a small fraction of the population that is treated and that for every value of the covariates $X$ there is at least a small chance that the unit is not treated. It's called common support when it's a propensity score but it's just about the distribution of treatment and control across values of $X$. 

\end{frame}

\begin{frame}{Estimating DD with Assumptions 1-3}

\begin{itemize}
\item Assumptions 1-3 gives us a couple of options of estimating the DiD
\item We can either use the outcome regression (OR) approach of Heckman, et al 1997
\item Or we can use the inverse probability weighting (IPW) approach of Abadie (2005)
\end{itemize}

\end{frame}


\begin{frame}{Outcome regression}

This is the Heckman, et al. (1997) approach where the outcome evolution is modeled with a regression

\bigskip

\begin{eqnarray*}
\widehat{\delta}^{OR} = \overline{Y}_{1,1} - \bigg [ \overline{Y}_{1,0} + \frac{1}{n^T} \sum_{i|D_i=1} ( \widehat{\mu}_{0,1}(X_i) - \widehat{\mu}_{0,0}(X_i)) \bigg ]
\end{eqnarray*}

where $\overline{Y}$ is the sample average of $Y$ among units in the treatment group at time $t$ and $\widehat{\mu}(X)$ is an estimator of the true, but unknown, $m_{d,t}(X)$ which is by definition equal to $E[Y_t|D=d,X=x]$.

\end{frame}

\begin{frame}{Inverse probability weighting}

This is the Abadie (2005) approach where we use weighting

\begin{eqnarray*}
\widehat{\delta}^{ipw} = \frac{1}{E_N[D]} E \bigg [ \frac{D-\widehat{p}(X)}{1-\widehat{p}(X)} (Y_1-Y_0) \bigg ]
\end{eqnarray*}

where $\widehat{p}(X)$ is an estimator for the true propensity score. Reduces the dimensionality of $X$ into a single scalar.

\end{frame}

\begin{frame}{These models cannot be ranked}

\begin{itemize}
\item Outcome regression needs $\widehat{\mu}(X)$ to be correctly specified, whereas
\item Inverse probability weighting needs $\widehat{p}(X)$ to be correctly specified
\item It's hard to ``rank'' these two in practice with regards to model misspecification because each is inconsistent when their own models are misspecified
\item Well why don't we just use TWFE?  I've never heard anyone complain about including covariates in TWFE and I've been doing it my entire adult life, so we're good right?
\item Depends on if you want to assume three more things. 
\end{itemize}

\end{frame}

\begin{frame}{TWFE}

Here's the TWFE specification:

\begin{eqnarray*}
Y_{it} = \alpha_1  + \alpha_2 T_t + \alpha_3 D_i +  \delta (T_i \times D_t)  + \varepsilon_{it}
\end{eqnarray*}

\bigskip

Just add in covariates then right?

\begin{eqnarray*}
Y_{it} = \alpha_1  + \alpha_2 T_t + \alpha_3 D_i  + \delta (T_i \times D_t) + \theta \cdot X_{it} + \varepsilon_{it}
\end{eqnarray*}

Sure! If you're willing to impose three \emph{more} assumptions

\end{frame}




\begin{frame}{Decomposing TWFE with covariates}

TWFE places restrictions on the DGP. Previous TWFE regression under assumptions 1-3 implies the following:

\bigskip

\begin{eqnarray*}
E[Y^1_1|D=1,X] = \alpha_1 + \alpha_2 + \alpha_3 + \delta + \theta X
\end{eqnarray*}

\bigskip

Conditional parallel trends implies

\small
\begin{eqnarray*}
&&E[Y^0_{1} - Y^0_{0}|D=1,X]= E[Y^0_{1} - Y^0_{0}|D=0,X] \\
&&E[Y^0_{1}|D=1,X] - E[Y^0_{0}|D=1,X]= E[Y^0_{1}|D=0,X] - E[Y^0_{0}|D=0,X] \\
&&E[Y^0_{1}|D=1,X] = E[Y^0_{0}|D=1,X] + E[Y^0_{1}|D=0,X] - E[Y^0_{0}|D=0,X] \\
&&E[Y^0_{1}|D=1,X] = E[Y_{0}|D=1,X] + E[Y_{1}|D=0,X] - E[Y_{0}|D=0,X] \\
\end{eqnarray*}\normalsize Last line from the switching equation. This gives us:

\begin{eqnarray*}
E[Y^0_{1}|D=1,X] = \alpha_1  + \alpha_2 + \alpha_3 + \theta X
\end{eqnarray*}


\end{frame}

\begin{frame}{Collecting terms}

\begin{eqnarray*}
&&E[Y^1_1|D=1,X] = \alpha_1 + \alpha_2 + \alpha_3 + \delta + \theta_1 X \\
&&E[Y^0_{1}|D=1,X] = \alpha_1  + \alpha_2 + \alpha_3 + \theta_2 X \\
&&E[Y^1_1|D=1,X]  - E[Y^0_{1}|D=1,X]  \\
&&=(\alpha_1 + \alpha_2 + \alpha_3 + \delta + \theta_1 X) - ( \alpha_1  + \alpha_2 + \alpha_3 + \theta_2 X )\\
&&=\delta + (\theta_1 X - \theta_2 X)
\end{eqnarray*}

\bigskip

By allowing for the possibility that $\theta_1 X \neq \theta_2 X$, we open up the possibility of bias from TWFE which is zero under three additional assumptions.

\end{frame}

\begin{frame}{Assumption 4: Homogeneous treatment effects in X}


TWFE requires homogenous treatment effects in $X$ (i.e., the treatment effect is the same for all $X$)

\bigskip

If $X$ is sex, then effects are the same for males and females.

\bigskip

  If $X$ is continuous, like income, then the effect is the same whether someone makes \$1 or \$1 million.

\end{frame}

\begin{frame}{X-specific trends}

TWFE also places restrictions on covariate trends for the two groups too.  Take conditional expectations of our TWFE equation. 

\begin{eqnarray*}
E[Y_1|D=1] &=& \alpha_1 + \alpha_2 + \alpha_3 + \delta + \theta X_{11} \\
E[Y_0|D=1] &=& \alpha_1 + \alpha_3 + \theta X_{10} \\
E[Y_1|D=0] &=& \alpha_1 + \alpha_2 + \theta X_{01} \\
E[Y_0|D=0] &=& \alpha_1 + \theta X_{00}
\end{eqnarray*}


\end{frame}


\begin{frame}{X-specific trends}

Now take the DiD formula:

\begin{eqnarray*}
\delta^{DD} = &&\bigg ( (\alpha_1 + \alpha_2 + \alpha_3 + \delta + \theta X_{11} ) - (\alpha_1 + \alpha_3 + \theta X_{10} ) \bigg )- \\
&& \bigg ( (\alpha_1 + \alpha_2 + \theta X_{01}) - (\alpha_1 + \theta X_{00}) \bigg )
\end{eqnarray*}

\bigskip

Eliminating terms, we get:

\begin{eqnarray*}
\delta^{DD} = &&\delta + \\
&& (\theta X_{11} - \theta X_{10} ) - (\theta X_{01} - \theta X_{00} )
\end{eqnarray*}

\bigskip

Second line requires that trends in X for treatment group equal trends in X for control group.

\end{frame}


\begin{frame}{Assumption 5 and 6}

We need ``no X-specific trends'' for the treatment group (assumption 5) and comparison group (assumption 6)

\bigskip

\textbf{Intuition}: No X-specific trends means the evolution of potential outcome $Y^0$ is the same regardless of $X$. This would mean you cannot allow rich people to be on a different trend than poor people, for instance.

\bigskip

Without these six, in general TWFE will not identify ATT. 

\end{frame}

\begin{frame}{Why not both?}

\begin{itemize}
\item Let's review the problem.  What if you claim you need $X$ for conditional parallel trends?
\item You have three options:
	\begin{enumerate}
	\item Outcome regression (Heckman, et al. 1997) -- needs Assumptions 1-3
	\item Inverse probability weighting (Abadie 2005) -- needs Assumptions 1-3
	\item TWFE (everybody everywhere all the time) -- needs Assumptions 1-6
	\end{enumerate}
\item Problem is 1 and 2 need the models to be correctly specified
\item Doubly robust combines them to give us insurance; we now get two chances to be wrong, as opposed to just one
\item I'm going to only stick to the panel data expressions bc all repeated cross-section does is add in some terms (and I've not written up semiparametric bounds yet)
\end{itemize}

\end{frame}


\begin{frame}{Notation}

\begin{eqnarray*}
&&p(x): \text{propensity score model} \\
&& \Delta Y = Y_1 - Y_0 = Y_{post} - Y_{pre} \\
&& \mu_{d,\Delta} = \mu_{d,1}(X) - \mu_{d,0}(X), \text{ where } \mu(X) \text{ is a model for} \\
&& m_{d,t} = E[Y_t|D=d,X=x]
\end{eqnarray*}So that means $\mu_{0,\Delta}$ is just the control group's change in average $Y$ for each $X=x$

\end{frame}

\begin{frame}{Population DR DiD model for panel data}

\begin{eqnarray*}
\delta^{dr} = E \bigg [ \bigg ( \frac{D}{E[D]} -\frac{ \frac{p(X)(1-D)}{(1-p(X))} }{E \bigg [\frac{p(X)(1-D)}{(1-p(X))} \bigg ]} \bigg  )( \Delta Y - \mu_{0,\Delta}(X)) \bigg ]
\end{eqnarray*}

Notice how the model controls for $X$: you're weighting the adjusted outcomes using the propensity score

\bigskip

The reason you control for $X$ twice is because you don't know which model is right.  DR DiD frees you from making a choice without making you pay too much for it


\end{frame}

\begin{frame}{Efficiency}

\begin{itemize}
\item Authors exploit all the restrictions implied by the assumptions to construct semiparametric bounds
\item This is where the influence function comes in, which those who have studied the DID code closely may have noticed
\item One of the main results of the paper is that the DR DiD estimator is also DR for inference
\item Let's skip to Monte Carlos
\end{itemize}

\end{frame}

\begin{frame}{Monte Carlo details}

\begin{itemize}
\item Compare DR with TWFE, OR and IPW
\item Sample size is 1,000
\item 10,000 Monte Carlo experiments
\item Propensity score estimated with logit; OR estimated using linear specification
\end{itemize}

\end{frame}



\begin{frame}[plain]

\begin{table}[htbp]\centering
\scriptsize
\caption{Monte Carlo Simulations, DGP1, Both OR and Propensity score correct}
\centering
\begin{threeparttable}
\begin{tabular}{l*{5}{c}}
\toprule
\multicolumn{1}{l}{\textbf{}}&
\multicolumn{1}{c}{\textbf{Bias}}&
\multicolumn{1}{c}{\textbf{RMSE}}&
\multicolumn{1}{c}{\textbf{SE}}&
\multicolumn{1}{c}{\textbf{Coverage}}&
\multicolumn{1}{c}{\textbf{CI length}}\\
\midrule
TWFE & -20.9518 & 21.1227 & 2.5271 & 0.000 & 9.9061 \\
OR & -0.0012 & 0.1005 & 0.1010 & 0.9500 & 0.3960 \\
IPW & 0.0257 & 2.7743 & 2.6636 & 0.9518 & 10.4412 \\
DR & -0.0014 & 0.1059 & 0.1052 & 0.9473 & 0.4124 \\
\bottomrule
\end{tabular}
\end{threeparttable}
\end{table}

\end{frame}


\begin{frame}[plain]
	\begin{figure}
	\includegraphics[scale=0.25]{./lecture_includes/mc_dr_1.png}
	\end{figure}

\end{frame}


\begin{frame}[plain]

\begin{table}[htbp]\centering
\scriptsize
\caption{Monte Carlo Simulations, DGP4, Neither OR and Propensity score correct}
\centering
\begin{threeparttable}
\begin{tabular}{l*{5}{c}}
\toprule
\multicolumn{1}{l}{\textbf{}}&
\multicolumn{1}{c}{\textbf{Bias}}&
\multicolumn{1}{c}{\textbf{RMSE}}&
\multicolumn{1}{c}{\textbf{SE}}&
\multicolumn{1}{c}{\textbf{Coverage}}&
\multicolumn{1}{c}{\textbf{CI length}}\\
\midrule
TWFE & -16.3846 & 16.5383 & 3.6268 & 0.000 & 14.2169 \\
OR & -5.2045 & 5.3641 & 1.2890 & 0.0145 & 5.0531 \\
IPW & -1.0846 & 2.6557 & 2.3746 & 0.9487 & 9.3084 \\
DR & -3.1878 & 3.4544 & 1.2946 & 0.3076 & 5.0749 \\
\bottomrule
\end{tabular}
\end{threeparttable}
\end{table}

\end{frame}

\begin{frame}[plain]
	\begin{figure}
	\includegraphics[scale=0.12]{./lecture_includes/mc_dr_2.png}
	\end{figure}


\end{frame}


\begin{frame}{Code}

There is code in R and Stata
\begin{itemize}
\item Stata: \textbf{drdid}
\item R: \textbf{drdid}
\end{itemize}
\bigskip
Remember -- it's for 2x2 with covariates (i.e., one treatment group)

\end{frame}




\begin{frame}{Concluding remarks}

\begin{itemize}
\item These two papers mark a different approach than is often the case for applied researchers who simply estimate regression models and hope they recover ``reasonably weighted'' causal effects
\item These new DiD start with target parameter and identification then build estimation
\item TWFE, as it turns out, is not mostly harmless
\end{itemize}

\end{frame}










\end{document}
